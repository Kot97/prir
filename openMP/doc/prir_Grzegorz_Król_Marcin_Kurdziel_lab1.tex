\documentclass[12pt]{article}
\usepackage{polski}
\usepackage{bchart}
\usepackage{graphicx} 
\usepackage[utf8]{inputenc}
\title{Sprawozdanie z labolatorium 1}
\author{Grzegorz Król, Marcin Kurdziel}
\begin{document}
 	\maketitle
 	\section{Specyfikacja maszyny testowej}
 		\begin{itemize}
 			\item System operacyjny: Ubuntu 20.04.2 pracujące w maszynie wirtualnej Oracle VirtualBox
 			\item Procesor Intel: i5-2400 4x3,1GHz, turbo boost 3,4Ghz
 			\item Ram przydzielony maszynie wirtualnej: 4GB DDR3 1333 MHz 
 		\end{itemize}	
	\section{Algorytm binaryzacji adaptacyjnej Bradleya}
	
		\subsection{Implementacja algorytmu}
		Algorytm zaimplementowany został w postaci klasy adaptive\_thresholding posiadającej dwie metody:
			\begin{itemize}
				\item
				 cv::Mat run\_serial() zawiera nierównoległą implementację algorytmu
				\item
				 cv::Mat run\_openmp(std::size\_t threads\_count) 	zawiera zrównolegloną przy pomocy OpenMP implementację algorytmu wykonywaną przy pomocy threads\_count wątków.
			\end{itemize}
		
		\subsection{Sposób testowania}
		Po uruchomieniu programu załadowany zostaje do pamięci wskazany obraz, a następnie przy pomocy funkcji benchmark zostaje dwudziestokrotnie uruchomiona każda z dwóch implementacji algorytmu. Oczywiście w trakcie trwania testów nie jest używane standardowe wyjście tekstu. Program zwraca średni czas potrzebny na wykonanie nierównoległej oraz równoległej wersji algorytmu. Do testów wykorzystany został przedstawiony obraz o rozmiarach 22373x4561 pikseli.
		\begin{figure}[ht]
		\includegraphics[width=\textwidth] {city2.jpg}
		\end{figure}
		\subsection{Zebrane wyniki}	
		Poniższa tabela zawiera zebrane wyniki: \newline
		
		\begin{tabular}{|r|l|}  \hline 
			serial		&	7,88591 s \\
			1 thread	&	7,45781 s \\
			2 threads	&	6,37856 s \\
			3 threads	&	6,00728 s \\
			4 threads	&	5,79122 s \\
			5 threads	&	5,93873 s \\
			6 threads	&	5,85136 s \\
			7 threads	&	5,93159 s \\
			8 threads	&	5,82764 s \\
			\hline
		\end{tabular}\newline 
		
		Poniższy wykres prezentuje dane tabeli w graficznej formie:	\newline
		\begin{bchart}[step=1, max=8, unit=s]
			\bcbar[text=serial]   {7.88591}
			\bcbar[text=1 thread] {7.45781}
			\bcbar[text=2 threads]{6.37856}
			\bcbar[text=3 threads]{6.00728}
			\bcbar[text=4 threads]{5.79122}
			\bcbar[text=5 threads]{5.93873}
			\bcbar[text=6 threads]{5.85136}
			\bcbar[text=7 threads]{5.93159}
			\bcbar[text=8 threads]{5.82764}
		\end{bchart}
	
	\section{Algorytm "Disarium Number"}
		\subsection{Implementacja algorytmu}
				Algorytm zaimplementowany został w postaci dwóch metod jednej zaierającej implementacje niezrównolegloną oraz drugiej zawierającej wersję wykorzystującą OpenMP
		\subsection{Sposób testowania}
				Podobnie jak i w przypadku poprzedniego algorytmu i tym razem użyta została funkcja benchmark jednak tym razem wykonująca 1000 powtórzeń algorytmu.
		\subsection{Zebrane wyniki}	
				Poniższa tabela zawiera zebrane wyniki: \newline
		\begin{tabular}{|r|l|} \hline
			serial		&	0,0517572 s \\
			1 thread	&	0,0488780 s \\
			2 threads	&	0,0250712 s \\
			3 threads	&	0,0180976 s \\
			4 threads	&	0,0138043 s \\
			5 threads	&	0,0194655 s \\
			6 threads	&	0,0172609 s \\
			7 threads	&	0,0151024 s \\
			8 threads	&	0,0139309 s \\
			\hline
		\end{tabular}\newline 
		
		Poniższy wykres prezentuje dane tabeli w graficznej formie:	\newline
		\begin{bchart}[step=0.03, max=0.06, unit=s]
			\bcbar[text=serial]   {0.0517572}
			\bcbar[text=1 thread] {0.0488780}
			\bcbar[text=2 threads]{0.0250712}
			\bcbar[text=3 threads]{0.0180976}
			\bcbar[text=4 threads]{0.01380432}
			\bcbar[text=5 threads]{0.0194655}
			\bcbar[text=6 threads]{0.0172609}
			\bcbar[text=7 threads]{0.0151024}
			\bcbar[text=8 threads]{0.0139309}
		\end{bchart}	
\end{document}